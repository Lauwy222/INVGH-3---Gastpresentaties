\section{Verklarende Woordenlijst}
%\addcontentsline{toc}{section}{Verklarende Woordenlijst}
\printglossaries

\newglossaryentry{ect.}{
  name = {ect.},
  description = {De betekenis van de Latijnse combinatie et cetera is 'enzovoort'. Voor et cetera wordt de afkorting etc.}
}

\newglossaryentry{DC}{
  name = {DC},
  description = {Direct Current (DC) is een elektrische stroom die in één richting vloeit, in tegenstelling tot wisselstroom (AC) waarin de stroomrichting periodiek omkeert.}
}

\newglossaryentry{gelijkstroomdistributiesysteem}{
  name = {gelijkstroomdistributiesysteem},
  description = {Een gelijkstroomdistributiesysteem verwijst naar een elektrisch distributiesysteem dat gebruikmaakt van gelijkstroom (DC) in plaats van wisselstroom (AC) om elektrische energie te transporteren en te verdelen.}
}

\newglossaryentry{DC-microgrids}{
  name = {DC-microgrids},
  description = {DC-microgrids zijn kleine, zelfvoorzienende elektrische netwerken die gelijkstroom (DC) gebruiken om energie lokaal op te wekken, op te slaan en te distribueren, vaak voor specifieke toepassingen of gebieden.}
}

\newglossaryentry{DC-netwerken}{
  name = {DC-netwerken},
  description = {DC-netwerken verwijzen naar elektrische netwerken die gebaseerd zijn op gelijkstroom (DC) in plaats van wisselstroom (AC) om elektriciteit te transporteren en te verdelen.}
}
