\subsection{Bezoek Siemens}
\subsubsection{Wat heb je gezien en gedaan tijdens het bezoek?}

Tijdens het bezoek aan Siemens in Den Haag heb ik verschillende interessante aspecten van het bedrijf en haar activiteiten ontdekt. Het kantoor heeft een moderne uitstraling en is voorzien van laadpalen die de focus leggen op Siemens Energy en smart infrastructure. Wij werden ontvangen door Marianne van Ruyven, een enthousiaste specialist in talentacquisitie met uitgebreide ervaring(drieentwintig jaar) bij Siemens, de introductie die ze gaf was informatief.

Een hoogtepunt van het bezoek was de rondleiding in de "Smart Infrastructure Experience Room". Hier kreeg ik inzicht in Siemens-producten, met name de groep- en zekeringsschakelaars die communiceren via het Internet of Things (IoT) met behulp van het MindSphere-cloudgebaseerde open IoT-besturingssysteem. Dit systeem, met toegang tot AWS-cloudservices, biedt connectiviteit, geavanceerde analyses en een partner-ecosysteem voor industriële oplossingen.

Verder heb ik geleerd over de Hunt-applicatie en Mindsphere, die het mogelijk maken om gegevens te verzamelen en te analyseren om problemen te identificeren en ongeplande downtime te minimaliseren in de industrie. In de "Experience Room" kwam ook het concept van Industrial Internet of Things (IIoT) aan bod, waar digitale simulaties en onderzoek naar optimalisatie worden uitgevoerd.

\subsubsection{Omschrijf jouw persoonlijke indrukken van het bezoek en de presentaties.}

Mijn persoonlijke indrukken van het bezoek zijn positief. De interactie met Marianne van Ruyven gaf een menselijk tintje aan het bedrijf, terwijl de presentaties in de "Smart Infrastructure Experience Room" en de informatie over MindSphere en IIoT me een dieper inzicht gaven in de geavanceerde technologieën die Siemens toepast.

De presentaties waren boeiend en interactief, met praktische voorbeelden zoals het spelen met het grote scherm om veranderingen te zien. De focus op digitalisering, IoT en IIoT laten zien hoe Siemens zich positioneert als een innovatieve speler in de industrie. Siemens veranderd steeds meer van een hardware naar software bedrijf.

\subsubsection{Zou het bedrijf een omgeving zijn waar jij een stage zou willen lopen of zou willen werken? Waarom wel of waarom niet?}

Siemens zou een bedrijf zijn waar ik stage zou willen lopen omdat het een stimulerende omgeving te bieden heeft voor iemand met interesse in technologie, digitalisering en IoT. De moderniteit van het kantoor, het verwelkomende ontvangst en de geavanceerde presentaties wekken de indruk dat Siemens toonaangevend is in de industrie.

Het gebruik van technologieën zoals MindSphere en de focus op IIoT wijzen op dat Siemens een innovatieve speler is in de sector. Als elektrotechniek student zou een stage bij Siemens een waardevolle kans kunnen bieden om praktijkervaring op te doen in de toepassing van IIoT en digitale transformatie. 

\href{https://jobs.siemens.com/careers/job/563156116443106?microsite=siemens.com}{Dit is een voorbeeld vacature voor een mogelijke stage opdracht.}