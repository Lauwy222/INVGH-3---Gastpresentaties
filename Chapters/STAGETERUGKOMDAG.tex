\section{Stageterugkomdag: Verslag en Toekomstige Stappen}

Tijdens de stageterugkomdag hebben we de mogelijkheid gehad om te luisteren naar de ervaringen van huidige stagiaires, die in presentaties meer hebben verteld over verschillende aspecten van hun stage. Hier volgt een verslag van deze dag en een bespreking van relevante punten:

\subsection{Sessies en Presentaties}

We hebben deelgenomen aan verschillende sessies en presentaties tijdens de stageterugkomdag. De keuze voor deze specifieke presentaties was gebaseerd op:

\begin{itemize}
    \item \textbf{Bedrijfsinformatie:} Vestigingsplaats, aantal collega's, belangrijkste activiteiten, en andere relevante details van het bedrijf.
    \item \textbf{Opdracht:} Achtergronden, aanpak, resultaten en andere informatie met betrekking tot de stageopdracht.
    \item \textbf{Arbeidsomstandigheden:} Zelfstandigheid, hulp, onkostenvergoeding, en andere aspecten die de werkomstandigheden beïnvloeden.
\end{itemize}

We hebben specifiek voor deze presentaties gekozen omdat ze relevant waren voor onze eigen zoektocht naar een geschikte stageplaats en een breder inzicht gaven in de diversiteit van stage-ervaringen.

\subsection{Relevante Informatie voor Zoektocht}

De verkregen informatie tijdens de stageterugkomdag heeft ons geholpen bij het verder vormgeven van onze zoektocht naar een geschikte stageplaats. De volgende aspecten zijn hierbij van belang:

\begin{itemize}
    \item \textbf{Bedrijfsprofiel:} Het begrijpen van de bedrijfscultuur en activiteiten om te bepalen of het bij onze verwachtingen past.
    \item \textbf{Opdrachtkenmerken:} Het analyseren van verschillende opdrachten om te bepalen welke het beste aansluit bij onze interesses en vaardigheden.
    \item \textbf{Arbeidsomstandigheden:} Het begrijpen van de werkomstandigheden om een inschatting te maken van de omgeving waarin we zouden werken.
\end{itemize}

\subsection{Toekomstige Stappen}

Om nu verder te gaan in onze zoektocht naar een geschikte stageplaats, hebben we een plan opgesteld met de volgende stappen:

\begin{enumerate}
    \item \textbf{Onderzoek:} Verdiepen in bedrijven en organisaties die overeenkomen met onze interessegebieden.
    \item \textbf{Sollicitaties:} Actief solliciteren naar stageplaatsen die aansluiten bij onze doelen.
    \item \textbf{Netwerken:} Gebruikmaken van professionele netwerken om informatie te verzamelen en mogelijkheden te ontdekken.
    \item \textbf{CV en Sollicitatiebrief:} Opstellen van een overtuigend CV en sollicitatiebrief.
    \item \textbf{Voorbereiding op Gesprekken:} Zich voorbereiden op sollicitatiegesprekken door mogelijke vragen en antwoorden te oefenen.
\end{enumerate}

Ook als we niet direct een stageplaats vinden, hebben we een tijdspad opgesteld dat aangeeft hoe we de periode naar de stage toe zullen invullen. Dit omvat activiteiten zoals het ontwikkelen van vaardigheden, het volgen van relevante cursussen en het blijven verkennen van mogelijkheden in het vakgebied.
