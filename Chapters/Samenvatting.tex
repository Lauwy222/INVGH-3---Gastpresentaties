% \section{Samenvatting}
% \subsection{Controller}
% In dit project is een op afstand bestuurbare auto ontwikkeld met behulp van een controller en een Bluetooth-module. De controller bestaat uit verschillende componenten, waaronder een TFT-display met een ili9341 IC, twee KY-023 joysticks en een krachtige microcontroller genaamd ESP32 D1 MINI, die draadloze verbindingen kan maken en complexe taken kan uitvoeren. Na het uitlezen van de waarden van de joysticks worden deze gekalibreerd en omgezet in een serieel formaat dat verzonden kan worden via de RF433MHZ-transceiver. De combinatie van RF433MHZ en de joysticks zorgt voor een betrouwbare en responsieve verbinding tussen de controller en de auto, waardoor de auto met precisie in elke richting kan worden bestuurd.

% \subsection{Motoraansturing}
% Om de vier motoren van de zelfrijdende auto aan te sturen, wordt gebruik gemaakt van een motorshield dat bestaat uit een 74HC595N shift-register en twee L293D H-brug motor drivers. Het shift-register creëert 8-bits extra digitale uitgangen zonder extra pinnen op de microcontroller te gebruiken en wordt gebruikt om gegevens naar de chip te sturen. De uitgangslatches van het shift-register sturen een positief of negatief signaal naar de L293D H-brug motor drivers, waarmee de motoren worden aangestuurd. Om een DC-motor in één richting te laten draaien, moeten de logische waarden op IN1 en IN2 voor motor 1 respectievelijk op HIGH en LOW worden gezet. De maximale PWM-waarde voor de motoren mag niet hoger zijn dan 220 om doorbranden van de motoren te voorkomen. Het is belangrijk om de stroomsterkte van de motor en het voltage van de voeding goed te controleren en af te stemmen op de specificaties van de L293D, die een maximale stroom van 600 mA per kanaal kan leveren en is uitgerust met ingebouwde beveiligingsfuncties, zoals thermische beveiliging uitschakeling en bescherming tegen kortsluiting, wat het veilig en betrouwbaar maakt om te gebruiken in diverse toepassingen.

% \subsection{Sensoren}
% Voor de zelfrijdende auto zijn sensoren gebruikt om objecten te detecteren en afstanden en oriëntatie te meten. Er zijn infrarood sensoren gebruikt voor het detecteren van objecten van 2 tot 30 centimeter en ultrasone sensoren voor het meten van afstanden tot 4 meter. Daarnaast is er een 9-assige oriëntatiesensor, genaamd de BNO055, die wordt gebruikt om de oriëntatie van de auto in de ruimte te meten. Voor de eisen van het project zijn er vier extra infrarood sensoren geplaatst, zodat de auto makkelijker door zijn omgeving kan manoeuvreren. De infrarood sensoren en de ultrasone sensoren zijn gepositioneerd aan de voorkant, achterkant en zijkanten van de auto. De sensoren kunnen worden aangestuurd via een protocol waarbij verschillende configuratieparameters kunnen worden ingesteld.