\subsection{Raoul}
\subsubsection{Stage 1: Boutronic}
\paragraph{Achtergrondinformatie van de Gastspreker}
Raoul heeft zijn stage gelopen bij Boutronic, een bedrijf opgericht in 1996 dat gespecialiseerd is in elektronica en een team heeft van 5 werknemers. Boutronic beschikt over eigen SMD pick\&place- en soldeertechnologie. Tijdens zijn stage werkte Raoul aan het ontwerpen van een PCB voor een vier voudige stroommeter en maakte hierbij gebruik van de STM4030, samen met het ontwikkelen van gerelateerde componenten zoals een scherm en een debugger circuit.

\paragraph{Wat maakte je nieuwsgierig in zijn/haar verhaal?}
De nieuwsgierigheid werd gewekt door de technische details en uitdagingen van het PCB-ontwerp en het praktisch toepassen van de STM4030 voor real-time stroommeting.

\paragraph{Welke competenties herken je in het verhaal van de gastspreker?}
De competenties die naar voren komen in Raoul zijn verhaal zijn technisch ontwerp, probleemoplossing, zelfmanagement en aanpassingsvermogen.

\paragraph{Ga na welke kennis, vaardigheden en houding je moet hebben om deze competenties goed uit te kunnen voeren.}
Voor deze competenties zijn gedetailleerde kennis van elektronische circuits, vaardigheid in PCB-ontwerp, kritisch denken, en een proactieve houding ten aanzien van problemen en onverwachte hindernissen noodzakelijk.

% \paragraph{Welke vraag/vragen heb je de gastspreker gesteld en wat was het antwoord?}
% Een passende vraag voor Raoul zou kunnen zijn: "Hoe ben je omgegaan met de afwezigheid van je stagebegeleider en hoe heeft dit je werk beïnvloed?" Een mogelijke reactie zou kunnen zijn: "Het was een uitdaging waar ik veel van heb geleerd over het belang van communicatie en zelfstandigheid. Ik had meer initiatief moeten nemen om contact op te nemen met mijn school toen bleek dat de begeleiding tekortschoot."

\paragraph{Wat is je het meest bijgebleven?}
Het meest memorabele aspect van Raoul zijn stage-ervaring is het belang van adequate begeleiding en de noodzaak om proactief te blijven communiceren met alle partijen (zowel stagebedrijf als opleiding) om stageproblemen te voorkomen.
