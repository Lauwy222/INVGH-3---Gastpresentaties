\subsection{Rong}
\subsubsection{Stage 1 en 2: ABB stage}
\paragraph{Achtergrondinformatie van de Gastspreker}
Rong heeft zijn stage gelopen bij ABB, binnen de Distribution Solutions-afdeling van Elektrotechniek in Rotterdam. Hij heeft gewerkt aan een project dat zich richt op de implementatie en het testen van kabelbeveiligingssystemen. ABB is een groot bedrijf, en Rong was onderdeel van een team dat zich bezighoudt met elektriciteitsdistributie van middenspanning naar laagspanning.

\paragraph{Wat maakte je nieuwsgierig in zijn/haar verhaal?}
Het gebruik van het Line differential protection systeem en de mogelijke implementatie van 5G technologie binnen het bedrijf heeft mijn aandacht getrokken, evenals Rongs betrokkenheid bij de innovatieve RD615 softwarematige simulatie.

\paragraph{Welke competenties herken je in het verhaal van de gastspreker?}
Rong demonstreert competenties als innovatieve technische probleemoplossing, productontwikkeling, technisch inzicht en zelfstandigheid.

\paragraph{Ga na welke kennis, vaardigheden en houding je moet hebben om deze competenties goed uit te kunnen voeren.}
Voor deze taken zijn kennis van elektrotechnische principes, vaardigheden in het werken met gespecialiseerde technische software, innovatievermogen, en een zelfstandige, proactieve werkhouding van belang.

% \paragraph{Welke vraag/vragen heb je de gastspreker gesteld en wat was het antwoord?}
% Een typische vraag voor Rong zou kunnen zijn: "Wat waren de grootste uitdagingen bij het samenbrengen van technologie en praktische toepassing in je stageproject, en hoe heb je deze aangepakt?" Rong zou kunnen antwoorden dat het omzetten van theorie naar praktische toepassingen altijd uitdagend is, maar door zelfstandig te werken en actief oplossingen te zoeken kon hij deze uitdagingen succesvol aanpakken.

\paragraph{Wat is je het meest bijgebleven?}
Wat het meest indruk heeft gemaakt, is Rong zijn inzicht dat je tijdens een stage ontdekt hoeveel je nog niet weet en hoeveel je ter plekke moet leren, benadrukkend dat de beroepspraktijk vaak een diepere leerervaring biedt dan theoretisch onderwijs.