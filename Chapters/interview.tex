\section{Interview met een Stagiair of Afgestudeerde in de Elektrotechnische Sector}
Tijdens dit deel van het verslag heb ik een interview gehouden met een stagiair of afgestudeerde in de branche waarin ik geïnteresseerd ben.

\subsection{Voorbereiding en Onderzoek}

Voorafgaand aan het interview heb ik mij grondig voorbereid door een vragenlijst op te stellen en onderzoek te doen naar het bedrijf of de organisatie waar de geïnterviewde werkzaam is. Voorbeeld vragen zijn:

\begin{itemize}
    \item \textbf{Belangrijkste werkzaamheden:} Wat zijn de dagelijkse taken en verantwoordelijkheden van de geïnterviewde?
    \item \textbf{Leuke en minder leuke kanten aan het werk:} Welke aspecten van het werk worden als positief ervaren en welke als uitdagend of minder leuk?
    \item \textbf{Vereiste vaardigheden en kennis:} Welke specifieke vaardigheden en kennis zijn cruciaal voor de functie?
    \item \textbf{Toepassing van opleidingskennis:} Hoe heeft de geïnterviewde de opgedane kennis en vaardigheden uit de opleiding toegepast in zijn/zijn werk?
    \item \textbf{Overige vragen:} ik heb ook enkele aanvullende vragen bedacht om een dieper inzicht te krijgen in de professionele ervaring van de geïnterviewde.
\end{itemize}

\subsection{Interview Resultaten}

Ik heb een afspraak gemaakt met de geïnterviewde en het interview gehouden volgens de opgestelde vragenlijst. De resultaten van het interview zijn als volgt:


% \subsubsection*{Vraag 1: Waarom heb je gekohijn voor een stage in binnen ABB en wat trok je specifiek aan in dit bedrijf?}
% \textit{Antwoord: }
% Kwam ABB tegen bij een bedrijfsmarkt en leek interessant. Bij twee bedrijven gesolliciteerd. Was direct raak bij ABB, interessante opdracht en veel vrijheid.

% \subsubsection*{Vraag 2: Kun je vertellen over je stage- of werkervaring bij ABB?}
% \textit{Antwoord: } 
% Veel onderzoek en documenten doorspitten voor informatie en dehij toepassen op jouw project. Wordt veel meegenomen in het werkveld met andere collegas en leert hier veel van. Laatst een installatie op high rating voltage, kwamen veel dingen bij kijken en verschillende componenten checken of die werken op high voltage. Veel datasheetslehijn. Staat altijd in bedrijf. Bepaalde componetnen waren niet compatabiel met de systemen.  Klant beslist dan. Vooral advies geven. Was mee bij warmte pomp installatie en deed metingen en testen. 
% Heeft hijlf beveilegingsrelais project gedaan. 

% \subsubsection*{Vraag 3: Wat zijn volgens jou de belangrijkste werkzaamheden en hoe zien jouw dagelijkse taken eruit?}
% \textit{Antwoord: } 
% Belangrijkste werkzaamheden zijn werken aan je eigen project. steeds wat verder komen. Dehij opdracht was een vage opdracht voor een prototype om te kijken of dit haalbaar was voor een student. Je start een project van begin en daardoor zie je alle fases van een project. doet twee kwartalen bij ABB stage. 

% \subsubsection*{Vraag 4: Zijn er aspecten van het werk die je als bijzonder uitdagend of interessant beschouwt? Zo ja, welke en waarom?}
% \textit{Antwoord: } 
% Je komt er dan pas achter hoe breed het bedrijf eigenlijk is. ABB heeft namelijk gigantisch veel afdelingen. Veel doorgroeimogelijkheden en veel mogelijkheden. 

% \subsubsection*{Vraag 5: Welke vaardigheden en kennis heb je ontwikkeld tijdens jouw stage, en hoe heb je dehij ontwikkeld?}
% \textit{Antwoord: } 
% Je leert veel meer hijlfstandigheid en Rong heef t heel veel leren onderzoeken. Rong leest nu makkelijk 600 paginas doorlehijn. Duurde ongeveer twee weken. ish 80 uur. Je leert ook over sociale vaardigheid ontwikkelen als je ergens niet achter kan komen. Competentie realiseren is wel lastig. 

% \subsubsection*{Vraag 6: Hoe heb je de kennis en vaardigheden die je tijdens je opleiding hebt opgedaan toegepast in je huidige stage?}
% \textit{Antwoord: } 
% Onderzoek doen leer je bij project tijdens je eerste twee jaren van de opleidingin. En je leert de basiskennis toepassen in het werkveld. Meeste leer je bij je stage. Het inzich wat je leert tijdens je studie pas je toe tijdens je stage.

% \subsubsection*{Vraag 7: Wat voor impact heeft jouw stage bij ABB?}
% \textit{Antwoord: } 
% Rong zijn product bestaat nog niet en is onderdeel van zijn project. Shell moet een groot vernieuwingsproject en daarbinnen komt dit test product goed toegepast. Er hoeft dus minder tijd vrijgemaakt worden bij ABB. Bij grote projecten kan je niet tegelijk installeren in tegenstelling tot kleine installaties. Rong heeft een klein steentje bijgedragen binnen ABB. Zijn product wordt hierna door ontwikkeld. 

% \subsubsection*{Vraag 8: Waarom heb je besloten om twee kwartalen in plaats van één kwartaal aan een specifieke opdracht te werken en wat zijn de voordelen hiervan?}
% \textit{Antwoord: } 
% Twee kwartalen want het project was dusdanig groot. en je hebt twee stageplekken als geregeld. Je leert meer en meer ontwikkelen ipv in 1 kwartaal. 


% \subsubsection*{Vraag 9: Hoe zie je jehijlf groeien in dit vakgebied op de lange termijn en overweeg je een werknemer te worden van ABB?}
% \textit{Antwoord: } 
% Overveegt om werknemer te worden bij ABB maar niet dehij afdeling. Liever micro electronica en robotica. Dit is meer elektro infrastructuur. Wilt graag in een groot groeiende sector werken.

% \subsubsection*{Vraag 10: Wat heb je exact ontwikkeld tijdens jouw stage en wat zijn de toekomstplannen voor kwartaal twee?}
% \textit{Antwoord: } 
% Het product is een testbox voor lijndiffrentiaal beveileging.
% Beveiliging over een node. Het is een verbinding waar stroom doorheen loopt. en de stroom moet in verhouding gelijk zijn. Beveiliging zorgt ervoor dat als de stroom niet in verhouding is is dat als iets niet in verhouding is dat het systeem wordt uitgeschakeld.
% Nadeel is dat aan beide punten een installatie moet plaatsen voordat men kan testen.
% Rong zijn product is dat er maar 1 beveiliging geplaatst moet worden voor het testen. Dit zorgt ervoor dat je eerder kan testen en zien of er problemen zijn. Dit zorgt ervoor dat de projecten soepeler lopen en er veel minder kosten zijn. Het project is ontwikkelen van de testkoffer waarin je dat kan testen, vanaf analyse tot aan beheren. Alle competenties worden hierin ontwikkeld. Aan het einde van het project worden toekomst plannen geschreven. Vandaar dat er twee kwartalen zijn genomen zodat er veel competenties ontwikkeld kunnen worden
\subsubsection*{Vraag 1: Waarom heb je gekohijn voor een stage binnen ABB en wat trok je specifiek aan in dit bedrijf?}

\textbf{Antwoord:} Rong Jin kwam een vertegenvoordiger van ABB tegen tijdens een bedrijfsmarkt en vond het interessant. Hij solliciteerde bij twee bedrijven en werd direct aangenomen bij ABB vanwege een boeiende opdracht en veel vrijheid. Wat zijn specifiek aantrok, was de breedte van het bedrijf met tallohij afdelingen, wat hem veel doorgroeimogelijkheden in de toekomst bied.

\subsubsection*{Vraag 2: Kun je vertellen over je stage- of werkervaring bij ABB?}

\textbf{Antwoord:} Rong Jin heeft veel onderzoek gedaan, documenten doorgenomen en informatie toegepast op zijn project. hij werkte nauw samen met collega's, deed metingen en testen, en was betrokken bij installaties op hoogspanning. zijn eigen project omvatte de ontwikkeling van een testbox voor lijndifferentiaalbeveiliging. Daarnaast heeft hij ook gewerkt aan een beveiligingsrelaisproject en beveiligingsadvies aan klanten gegeven.

\subsubsection*{Vraag 3: Wat zijn volgens jou de belangrijkste werkzaamheden en hoe zien jouw dagelijkse taken eruit?}

\textbf{Antwoord:} De belangrijkste werkzaamheden van Rong Jin bestonden uit het werken aan zijn eigen project, waarbij hij steeds verder vorderde. De opdracht was aanvankelijk vaag, een prototype voor haalbaarheid, waardoor hij alle fases van een project doorliep. zijn dagelijkse taken omvatten onderzoek, projectwerk en betrokkenheid bij verschillende projecten binnen ABB.

\subsubsection*{Vraag 4: Zijn er aspecten van het werk die je als bijzonder uitdagend of interessant beschouwt? Zo ja, welke en waarom?}

\textbf{Antwoord:} Rong Jin vond het bijzonder uitdagend om de breedte van het bedrijf te ontdekken, met talloze afdelingen en doorgroeimogelijkheden. Het werk op hoogspanning, zoals installaties testen, componenten controleren en adviseren aan klanten, vond hij interessant vanwege de complexiteit en de impact ervan op grote projecten.

\subsubsection*{Vraag 5: Welke vaardigheden en kennis heb je ontwikkeld tijdens jouw stage, en hoe heb je dehij ontwikkeld?}

\textbf{Antwoord:} Tijdens de stage heeft Rong Jin hijlfstandigheid ontwikkeld en geleerd grondig onderzoek te doen. hij kon nu gemakkelijk 600 pagina's documentatie doornemen na ongeveer twee weken, wat zijn onderzoeksvaardigheden aanzienlijk verbeterde. Het ontwikkelen van sociale vaardigheden, vooral wanneer hij vastliep, was ook een belangrijke competentie.

\subsubsection*{Vraag 6: Hoe heb je de kennis en vaardigheden die je tijdens je opleiding hebt opgedaan toegepast in je huidige stage?}

\textbf{Antwoord:} Rong Jin paste de onderzoeksvaardigheden die hij in de eerste twee jaren van zijn opleiding had geleerd toe bij zijn project. Hij gebruikte ook de basiskennis van elektrotechniek en dit paste hij toe in de praktijk. De stage bood de mogelijkheid om de theoretische kennis toe te passen in real-world situaties.

\subsubsection*{Vraag 7: Wat voor impact heeft jouw stage bij ABB?}

\textbf{Antwoord:} Rong Jin's product, een testbox voor lijndifferentiaalbeveiliging, was een waardevolle bijdrage aan een innovatieproject van Shell. Het verminderde de tijd die nodig is voor commissioning, waardoor kosten werden bespaard. zijn product wordt nu doorontwikkeld, wat de efficiëntie van toekomstige projecten bij ABB zal verbeteren.

\subsubsection*{Vraag 8: Waarom heb je besloten om twee kwartalen in plaats van één kwartaal aan een specifieke opdracht te werken en wat zijn de voordelen hiervan?}

\textbf{Antwoord:} Rong Jin koos voor twee kwartalen vanwege de omvang van zijn project. Het gaf zijn de mogelijkheid om diepgaander te werken en meer vaardigheden te ontwikkelen dan bij twee verschillende stageplekken. Het verlengde tijdsbestek bood meer kansen voor groei en ontwikkeling.

\subsubsection*{Vraag 9: Heb je advies voor andere studenten die overwegen stage te lopen bij ABB of in een vergelijkbaar vakgebied?}

\textbf{Antwoord:} Rong Jin adviseert studenten om open-minded en nieuwsgierig te zijn. hij benadrukt het belang van het stellen van vragen en deel te nemen aan verschillende activiteiten binnen het bedrijf. Het opbouwen van een netwerk en het heb van een brede focus zijn essentieel voor een succesvolle stage bij ABB of een vergelijkbaar bedrijf.


\subsection{Persoonlijke Reflectie}
\textit{Bepaal voor jezelf of zijn/zijn functie jou aanspreekt. Waarom wel of waarom niet?}\\
De functie spreekt me over het algemeen aan vanwege het intrigerende karakter, maar het is niet de sector waarin ik mezelf zie werken. Hoewel ik de taken interessant vind, trekken andere vakgebieden mij meer aan. Mijn voorkeur gaat uit naar gebieden waar ik mijn passies en vaardigheden beter kan benutten. Desondanks waardeer ik de complexiteit van de functie en ben ik ervan overtuigd dat het voor de juiste persoon een boeiende uitdaging zou zijn.