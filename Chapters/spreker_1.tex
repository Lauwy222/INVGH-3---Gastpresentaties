\section{Gastspreker 1}
Bij deze spreker was ik niet aanwezig dus kan ik geen antwoord geven op devolgende vragen:
\begin{itemize}
    \item Achtergrondinformatie van de gastspreker: naam, leeftijd, vakgebied, functie, organisatie, ect.
    \item Wat maakte je nieuwsgierig in zijn/haar verhaal?
    \item Welke competenties herken je in het verhaal van de gastspreker?
    \item Ga na welke kennis, vaardigheden en houding je moet hebben om deze competenties goed uit te kunnen voeren.
    \item Welke vraag/vragen heb je de gastspreker gesteld en wat was het antwoord?
    \item Wat is je het meest bijgebleven?
\end{itemize}
\subsection{Wie is Laurens?}
Laurens MacKay van DC Opportunities lijkt een expert te zijn op het gebied van gelijkstroom (direct current, DC) microgrids en DC-distributienetwerken. Hij heeft een grote academische achtergrond in elektrische techniek en informatietechnologie en heeft zowel een bachelor als een mastergraad behaald aan ETH Zurich in Zwitserland. Tijdens zijn master's thesis begon hij in 2012 te werken aan DC-distributienetwerken. Later, aan de Technische Universiteit Delft in Nederland, behaalde hij zijn doctoraat in 2018, waarbij zijn proefschrift betrekking had op "Stappen naar het universele gelijkstroomdistributiesysteem" (Steps Towards the Universal Direct Current Distribution System).\cite{LaurensLINKEDIN}
\subsection{Wie is DC Opportunities?}
Het bedrijf DC Opportunities, opgericht door Dr. Laurens Mackay, richt zich op de ontwikkeling van technologieën en toepassingen voor DC-microgrids en DC-distributienetwerken. Een specifiek project waarbij DC Opportunities betrokken is, heet "The Green Village," waarin ze samenwerken met CityTec om prototypes te ontwikkelen en testen voor het opladen van elektrische voertuigen (EV's) met behulp van bestaande DC-netwerken of kabels die zijn aangelegd voor straatverlichting op DC. Het doel is om elektrisch opladen in de toekomst gemakkelijker en kostenefficiënter te maken. \cite{DCSITE} \cite{DCLINKEDIN} \cite{Innovation_Quarter}
\subsection{Samengevat}
Kortom, Laurens MacKay en DC Opportunities richten zich op de bevordering van DC-technologieën en het verbeteren van DC-microgrids en distributienetwerken, met als doel het efficiënter maken van elektrische systemen en het ondersteunen van groene energie-initiatieven zoals elektrisch vervoer.