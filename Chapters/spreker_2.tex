\subsection{Gastspreker 2}
\subsubsection{Achtergrondinformatie van de gastsprekers}
De gastsprekers, Jack Geurtsen en Rob Leijs, zijn beide professionals bij Huygen Installatie Adviseurs. Jack heeft een achtergrond in technische bedrijfskunde, hogere installatietechniek en een master in projectmanagement. Hij begon zijn loopbaan als monteur en klom op tot de positie van projectleider bij Huygen. Rob heeft een achtergrond in technische natuurkunde en heeft 10 jaar ervaring als projectleider bij een systeemintegrator voordat hij bij Huygen ging werken.
\subsubsection{Wat maakte je nieuwsgierig in zijn/haar verhaal?}
Het interessante aan hun verhaal is hoe ze beiden vanuit verschillende achtergronden en startposities zijn gegroeid naar leidinggevende rollen in Huygen Installatie Adviseurs. Jack begon als monteur en klom op tot projectleider, terwijl Rob een sterke technische achtergrond in natuurkunde had voordat hij projectleider werd. Deze diverse paden naar succes in dezelfde organisatie wekken nieuwsgierigheid op over hun persoonlijke ontwikkeling en de kansen die Huygen biedt.
\subsubsection{Welke competenties herken je in het verhaal van de gastspreker?}
In hun verhaal herken ik verschillende competenties:
\begin{itemize}
    \item \textbf{Deskundigheid:} Beide sprekers benadrukten het belang van deskundigheid en voortdurende kennisontwikkeling. Ze voeren onderzoek uit en zorgen voor altijd deskundig advies aan klanten.
    \item \textbf{Integrale benadering:} Ze legden uit dat Huygen zich richt op het bekijken van het gehele gebouw en niet alleen op specifieke producten, wat wijst op hun vermogen om integraal te denken.
    \item \textbf{Onafhankelijkheid:} Huygen streeft naar onafhankelijkheid en zoekt niet naar productgerichte oplossingen. Ze streven naar de beste oplossing, niet alleen de snelste.
    \item \textbf{Technische vaardigheden:} Het verhaal benadrukte technische vaardigheden in gebouwautomatisering, regeltechniek, software, en elektronische schema's.
    \item \textbf{Communicatieve vaardigheden:} Het belang van communicatieve vaardigheden bij het uitleggen van ontwerpen en realisaties aan klanten en teamleden werd benadrukt.
    
\end{itemize}
\subsubsection{Ga na welke kennis, vaardigheden en houding je moet hebben om deze competenties goed uit te kunnen voeren.}
Om de genoemde competenties goed uit te voeren, zijn de volgende aspecten cruciaal:
\begin{itemize}
    \item \textbf{Technische kennis:} Een diepgaande kennis van installatietechniek, automatiseringssystemen en regeltechnieken is essentieel.
    \item \textbf{Onderzoeksvaardigheden:} Het vermogen om onderzoek te doen naar nieuwe technologieën en methoden in de branche.
    \item \textbf{Integraal denken:} Het vermogen om problemen te benaderen vanuit een holistisch perspectief en te begrijpen hoe verschillende systemen samenwerken.
    \item \textbf{Onafhankelijkheid:} Het vermogen om objectieve, productonafhankelijke adviezen te geven.
    \item \textbf{Communicatievaardigheden:} Sterke communicatieve vaardigheden zijn nodig om complexe technische concepten aan niet-technische belanghebbenden uit te leggen.
\end{itemize}
\subsubsection{Welke vraag/vragen heb je de gastspreker gesteld en wat was het antwoord?}
Voor mij was alles in het verhaal duidelijk, ik heb dus geen vragen gesteld.
\subsubsection{Wat is je het meest bijgebleven?}
Het meest opmerkelijke aspect dat bijblijft, is het belang van deskundigheid, integraal denken en onafhankelijkheid in het werk van Huygen Installatie Adviseurs. Ze streven naar het bieden van de beste oplossingen voor klanten en zien het gehele gebouw als een systeem waarin alles met elkaar samenhangt. Dit benadrukt het belang van een diepgaande technische kennis en een holistische benadering in de branche. Ook benadrukken de gastsprekers het belang van voortdurende kennisontwikkeling en goede communicatieve vaardigheden, wat van cruciaal belang is in hun werk. Tot slot, de praktische tips voor studenten die op zoek zijn naar stages, zoals op tijd komen en werk doen dat je leuk vindt, zijn waardevolle inzichten voor iedereen die een carrière in dit vakgebied overweegt.