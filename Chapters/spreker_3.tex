\subsection{Gastspreker 3}
\subsubsection{Achtergrondinformatie van de gastspreker}
Florian werkt momenteel als Secundair Engineer bij Volker Energy Solutions. Zijn opleidingsachtergrond omvat maritieme studies, hoewel hij na een jaar besloot te stoppen en via YER geïnteresseerd raakte in een andere studierichting. Hij heeft zichzelf ontwikkeld tot een Elektrotechnisch Engineer en behaalde het certificaat in dit vakgebied. Florian benadrukt kernwaarden als veiligheid, duurzaamheid en integriteit in zijn werk. Momenteel is hij betrokken bij een interessant project met waterstofproductie bij Shell Holland Hydrogen.

\subsubsection{Wat maakte je nieuwsgierig in zijn verhaal?}
Het verhaal van Florian wekte mijn nieuwsgierigheid vanwege zijn opmerkelijke carrièreswitch van maritieme studies naar Elektrotechnisch Engineering via YER. Zijn betrokkenheid bij het Shell Holland Hydrogen-project, dat waterstofproductie omvat, was ook facinerend vanuit het oogpunt van duurzame energie.

\subsubsection{Welke competenties herken je in het verhaal van de gastspreker?}
Uit Florian's verhaal blijkt dat hij verschillende competenties heeft ontwikkeld, waaronder:
\begin{itemize}
    \item Technische vaardigheden in het opstellen van elektrische schema's en het uitvoeren van berekeningen, zoals kabelberekeningen.
    \item Sterke samenwerkingsvaardigheden, zoals blijkt uit zijn positieve samenwerking met YER.
    \item Het benadrukken van kernwaarden zoals veiligheid, duurzaamheid en integriteit, wat getuigt van sterke ethische en professionele competenties.
\end{itemize}

\subsubsection{Ga na welke kennis, vaardigheden en houding je moet hebben om deze competenties goed uit te kunnen voeren.}
Om de competenties van Florian goed uit te kunnen voeren, zijn de volgende aspecten cruciaal:
\begin{itemize}
    \item Diepgaande kennis van elektrotechniek, inclusief het begrijpen van verschillende elektrische schema's en het uitvoeren van berekeningen volgens normen zoals NEN1010.
    \item Technische vaardigheden om nauwkeurige elektrische schema's en berekeningen te kunnen maken.
    \item Vermogen om effectief te werken in kleinere teams en tegelijkertijd meerdere projecten te beheren.
    \item Sterke samenwerkingsvaardigheden en het vermogen om duidelijk te communiceren met teamleden en partners.
    \item Een sterke ethische houding met betrekking tot veiligheid, duurzaamheid en integriteit.
\end{itemize}

\subsubsection{Welke vraag/vragen heb je de gastspreker gesteld en wat was het antwoord?}
Voor mij was alles in het verhaal duidelijk, ik heb dus geen vragen gesteld.

\subsubsection{Wat is je het meest bijgebleven?}
Wat me het meest bijblijft uit het verhaal van Florian is zijn vermogen om zijn carrière om te buigen van maritieme studies naar Elektrotechnisch Engineering en zijn betrokkenheid bij het boeiende Shell Holland Hydrogen-project. Deze transitie bevestigd zijn flexibiliteit en vastberadenheid om nieuwe uitdagingen aan te gaan om zijn professionele ontwikkeling te bevorderen. Bovendien benadrukt zijn werk aan duurzame energieprojecten de groeiende relevantie van deze sector in onze samenleving.