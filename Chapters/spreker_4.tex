% Later in de les dus ik begin bij het Deel Delft Semiconductor’s first own product.
% Eerste product was/is een OPAMP(Operational Amplifier) Model is Low-TCVos Operational Amplifier)
% Er warden 10 miljard OPAMPS afgelopen jaar verkocht. Waarom hebben zij er 1 gemaakt, dicht bij ideale situatie. Bv, geen signaal in, wel signaal uit, dit is dus ruis dat geboost wordt. Dat is niet goed.
% Mogelijkheden voor ideale: 
% -	Alles wordt vooraf gemeten tijdens productie, en dan de correcties worden met een laser gedaan.
% -	Een filter toepassen.
% Wat heeft Delft Semiconductor gedaan:
% Moment na het packagen, de conducter regelen op twee temperaturen. Hierbij wordt de sensor gecalibreerd.
% Mijn vraag: Hoe gaat het calibreren van de chip? Op de chip zit een OTP(one time programmable) register. Als de ingangen kort worden gesloten kan de uitgang uitgelezen worden en kan de chip gecalibreerd worden, na het calibreren wordt het OTP fuse(polyweerstand) doorgebrand waardoor de chip vanaf dat moment alleen nog maar kan lezen wat de calibratie is en dat toepassen.
% Mijn vraag: Waar laat je de chips maken? Chips laten ze maken bij TMNC op 180nanometer. OTP is een ingekocht IP. Kwikpack is het package bedrijf.
% Chip naam is DS9604
% Mijn vraag: Hoe vindt je de fouten/ opgeblazen producten? Delft Semiconductor reproduceert de fouten in simulatie of laat het product opsturen naar Hengelo waar er thermische en/of opengemaakte chip fotos gemaakt worden. Hierin kan je de stromen zien bewegen.
% Door middel van een OPAMP lus kan je verschillende parameters doormeten.
% Wat voor studenten, PCB ontwikkelen of automatisering van meten van component of het automatisch kalibreren van de componenten. Mocht je een leuke opdracht zoeken voor stage, gewoon mailen en samen kunnen ze er uit komen. Delft Semiconductor is een team van 3 man nu.
% Voor ontwikkeling van de OPAMPS wordt een subsidie gegeven. Dat heb je loonheffing waardoor de werknemers goedkoper worden voor het bedrijf. Subsidie wordt gegeven door WBSO. Het product moet nieuw voor het bedrijf zijn, dan krijg je subsidie.
% Er is wel redelijk veel vraag naar Delft Semiconductor, maar het binnenkrijgen van klanten is lastig.
% Tip voor start-up: Linkedin veel mensen berichten en emailen. Een veel via via samenwerkingen. Het duurde hun 8 maanden voordat ze hun start-up draaiende hadden. 
\subsection{Spreker 4}
\subsubsection{Achtergrondinformatie van de gastspreker:}
Henk is werkzaam is bij Delft Semiconductor, hij introduceerde ons in de wereld van elektronica en halfgeleiders. Met een passie voor dit vakgebied speelt hij een cruciale rol binnen de organisatie, namelijk: mede-eigenaar.

\subsubsection{Wat maakte je nieuwsgierig in zijn/haar verhaal?}
Wat mijn nieuwsgierigheid wekte, was het moment waarin Henk begon te praten over Delft Semiconductor's eerste eigen product - een kalibreerbare Operational Amplifier (OPAMP). De keuze voor dit specifieke product en het streven naar een bijna ideale situatie in termen van ingangssignaalruis interesseerde mij. Het leek alsof hij een verhaal vertelde over het najagen van de ideale situatie in de huidige technologische wereld.

\subsubsection{Welke competenties herken je in het verhaal van de gastspreker?}
Terwijl de spreker sprak over het ontwerpen en produceren van elektronische componenten, werd duidelijk dat technische expertise hier de sleutel is. Zijn vermogen om problemen op te lossen, vooral in de context van fouten in chipproductie, bevestigde van een diepgaand begrip van het vakgebied.

\subsubsection{Welke kennis, vaardigheden en houding zijn nodig voor deze competenties?}
De gastspreker lijkt te beschikken over diepgaande kennis van elektronica, chipproductie en kalibratietechnieken. Zijn vaardigheden om simulaties uit te voeren, fouten te reproduceren en meetinstrumenten te gebruiken, werden benadrukt als essentieel in zijn vakgebied.

\subsubsection{Welke vraag/vragen heb je de gastspreker gesteld en wat was het antwoord?}
Tijdens de interactie met de gastspreker zijn er vragen gesteld die dieper ingingen op zijn werk bij Delft Semiconductor. Mijn eerste vraag betrof het specifieke proces van chipkalibratie. Het antwoord dat ik kreeg was het gebruik van een One Time Programmable (OTP) register, waarbij kortsluiting van de ingangen de basis vormt voor kalibratie. Vervolgens informeerde ik naar de productielocatie van de chips, waarbij ik te horen kreeg dat TMNC op de fabrikant is van deze geavanceerde chips. 


\subsubsection{Wat is je het meest bijgebleven?}
Van alle informatie was het vooral de innovatieve benadering van chipproductie die indruk op me maakte. Het gebruik van het concept OTP-register voor kalibratie toonden aan dat de spreker en zijn team niet bang zijn om grenzen te verleggen. Daarnaast bleef het belang van samenwerking en de inspirerende reis van een klein team binnen een start-up me sterk bij. De gastspreker deelde niet alleen kennis, maar ook de waardevolle levens/bedrijfs-lessen die hij heeft geleerd tijdens de acht maanden durende opbouw van Delft Semiconductor als start-up.
