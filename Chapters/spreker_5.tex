\subsection{Jochem} 
\subsubsection{Stage 1: Quantified} \paragraph{Kennis, vaardigheden en houding} 
Jochem heeft in zijn verhaal verschillende competenties getoond, zoals elektronica, hardwareontwerp, teamwerk, communicatie en creativiteit. Om deze competenties goed uit te kunnen voeren, moet je goede kennis hebben van elektronica en hardwareontwerp. Je moet ook in staat zijn om goed te werken met teams en goede communicatievaardigheden hebben. Verder moet je creatief zijn om oplossingen te vinden voor technische problemen.

% \paragraph{Vragen aan Jochem} 
% Ik heb Jochem gevraagd of hij ooit geconfronteerd is geweest met technische problemen tijdens zijn projecten en wat hij deed om deze op te lossen. Hij antwoordde dat hij vaak met zijn teamleden samenwerkte om oplossingen te vinden en dat hij ook graag naar andere bronnen zocht om meer informatie over het probleem te krijgen.

\subsubsection{Stage 2: DC lab} \paragraph{Kennis, vaardigheden en houding} 
In zijn tweede project heeft Jochem zijn vaardigheden in PCB-ontwerp verder verbeterd. Hij moest ook in staat zijn om goed te werken met teams en goede communicatievaardigheden hebben. Verder moet je creatief zijn om oplossingen te vinden voor technische problemen.

% \paragraph{Vragen aan Jochem} 
% Ik heb Jochem gevraagd of hij ooit geconfronteerd is geweest met technische problemen tijdens zijn projecten en wat hij deed om deze op te lossen. Hij antwoordde dat hij vaak met zijn teamleden samenwerkte om oplossingen te vinden en dat hij ook graag naar andere bronnen zocht om meer informatie over het probleem te krijgen.

\paragraph{Het meest bijgebleven} 
Het meest bijgebleven aan zijn verhaal is hoe Jochem zijn vaardigheden in elektronica kon ontwikkelen door deel te nemen aan verschillende projecten en hoe hij zijn creativiteit kon combineren met zijn technische kennis om oplossingen te vinden voor technische problemen.

\subsubsection{Stage 2: DC Lab}
\paragraph{Wat maakte je nieuwsgierig in zijn/haar verhaal?}
Ik was vooral geïntrigeerd door hoe Jochem het PCB-ontwerp onder de knie kreeg en hoe hij zijn creativiteit toepaste in de praktijk van hardwareontwikkeling.

\paragraph{Welke competenties herken je in het verhaal van de gastspreker?}
In deze tweede stage heeft Jochem zijn vaardigheden op het gebied van PCB-ontwerp en prototyping duidelijk geëtaleerd. Bovendien bleek uit zijn verhaal een diepe betrokkenheid bij teamcollaboratie en probleemoplossing.

\paragraph{Ga na welke kennis, vaardigheden en houding je moet hebben om deze competenties goed uit te kunnen voeren.}
Naast technische kennis van PCB-ontwerp en elektronica, moet men ook projectmanagementvaardigheden, aandacht voor detail en een bereidheid om voortdurend te leren hebben.

\paragraph{Welke vraag/vragen heb je de gastspreker gesteld en wat was het antwoord?}
Toen ik Jochem vroeg hoe hij het ontwerpproces aanpakte, benadrukte hij het belang van iteratieve ontwikkeling en het voortdurend testen van prototypes om de best mogelijke resultaten te bereiken.

\paragraph{Wat is je het meest bijgebleven?}
Het was indrukwekkend om te leren over Jochems toewijding aan zijn vakgebied en hoe hij voortdurend streeft naar verbetering en innovatie binnen de hardware-industrie.
