\section{Terugblik op Semester 1}
Het afgelopen kwartaal was een leerzame periode waarin ik verschillende ervaringen heb opgedaan en mezelf zowel academisch als persoonlijk heb ontwikkeld.

\subsection{Ervaring Semester 1}
Semester 1 was voor mij een erg fijne periode waarin ik actief bezig ben geweest met verschillende projecten. In mijn vrije tijd heb ik mij geïnteresseerd in het ontwikkelen van een eigen led controller, klussen aan mijn auto, voortzetten van mijn eigen bedrijf RedShot Productions, en deelname aan het Elektro social team. Over het algemeen ging het volgen van lessen goed, maar ik merkte dat de consistentie in het maken van huiswerk beter kon. Laatste cijfers hebben me enigszins gedemotiveerd, maar ik ben van plan alles bij te werken tijdens de kerstvakantie. Ik geloof sterk in het principe dat hard werken loont en zal me hierop blijven richten.

\subsection{Vakken}
De vakken die voorspoedig verliepen waren \textbf{UCPROG}, \textbf{INGVH}, en het \textbf{Project}. Deze vakken brachten veel plezier en succesvolle resultaten met zich mee. Aan de andere kant vond ik \textbf{Wiskel} en \textbf{Wisan} uitdagender. \textbf{Wiskel} test 1 verliep goed, maar test 2 niet. \textbf{Wisan} test 1 ging niet goed, en ik hoop dat \textbf{Wisan} test 2 betere resultaten zal opleveren.

\subsection{Successen}
Twee successen waar ik tevreden over ben:

\begin{enumerate}
    \item \textbf{Toename in doorzettingsvermogen en interesse:} Ik heb actief contact gelegd met professionals in het werkveld, wat mijn interesse in het vakgebied vergrootte en mijn doorzettingsvermogen versterkte.
    \item \textbf{Verbetering in LaTeX-vaardigheden:} Ik heb meer geleerd over LaTeX, wat mijn vermogen om verslagen te maken en presenteren aanzienlijk heeft verbeterd.
\end{enumerate}

\subsection{Leerpunten}
Twee leerpunten waar ik aan wil werken voor volgend semester:

\begin{enumerate}
    \item \textbf{Consistentie in huiswerk:} Ik wil meer aandacht besteden aan het consistent maken van huiswerk om een betere academische discipline te ontwikkelen.
    \item \textbf{Efficiënt gebruik van praktica-uren:} Ik wil proberen alle praktica tijdens de daarvoor bestemde uren af te ronden, om te voorkomen dat ik dit na schooltijd moet inhalen vanwege tijdgebrek.
\end{enumerate}

Deze terugblik helpt me niet alleen om mijn prestaties te evalueren, maar ook om gerichte doelen te stellen voor het komende semester.