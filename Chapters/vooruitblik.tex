\section{Vooruitblik op Blok 2}
\subsection{Studieplan voor Semester 2}

In semester 2 plan ik de volgende vakken te volgen:
\begin{itemize}
    \item \textbf{Project (jaar 2)}
    \item \textbf{INGVH1 Statistiek (jaar 1):} Voortzetten van het vak Statistiek.
    \item \textbf{UCPROG (jaar 1)}.
    \item \textbf{WISAN2 (jaar 1)}.
\end{itemize}

\subsection{Eventuele Belemmeringen}
Voor het vak Project zie ik mogelijk belemmeringen, met name op het gebied van praktische ervaring en het opdoen van kennis. Om dit aan te pakken, heb ik bronnen zoals IEEE geïdentificeerd waar ik aanvullende informatie kan verkrijgen. Bovendien ben ik van plan om samen te werken met medestudenten en hogerejaars studenten, en hun expertise te raadplegen wanneer ik iets niet begrijp.

\subsection{Leerdoel volgens SMART}
Mijn leerpunt vanuit de terugblik is dat ik consistenter huiswerk wil maken en me strikt wil houden aan mijn planning. Om dit te bereiken formuleer ik het leerdoel volgens het SMART-principe:

\begin{itemize}
    \item \textbf{Specifiek:} Consistenter huiswerk maken en me houden aan mijn planning.
    \item \textbf{Meetbaar:} Minimaal 90\% van het geplande huiswerk wordt op tijd voltooid.
    \item \textbf{Acceptabel:} Het doel is realistisch en haalbaar binnen mijn studiebelasting.
    \item \textbf{Relevant:} Het is relevant omdat het bijdraagt aan mijn academische discipline en consistentie.
    \item \textbf{Tijdgebonden:} Binnen dit semester wil ik dit doel bereiken en evalueren aan het einde van het semester.
\end{itemize}

Dit SMART-leerdoel zal mij helpen om mijn studie-efficiëntie te verbeteren en de ervaring van het studeren in semester 2 te optimaliseren.